\documentclass[12pt,twocolumn,oneside]{book}

\begin{document}

\newcommand{\reference}[3]{
        \textbf{#1:} See \texttt{#2}, the line starting with \texttt{#3}.
    }

\chapter*{Introduction}

The purpose of this document is to use Coq to go over the examples from
\emph{Category Theory for Scientists}, a book on Category Theory by David
Spivak at MIT that is designed to be accessible to non-mathematicians. Often it
can be easy to be lost in mathematical definitions. However, programming definitions
can often be easier: programs require a much greater degree of explicitness.

The language Coq is good for this purpose because its types correspond to those
of category theory quite effectively. Additionally, being dependently typed means
that it is not difficult to, for example, define the set of numbers less than $n$
as a type dependent on $n$.

In any case, this document is intended to be an index, taking the form of \emph{CT4S}
in terms of chapter, section, etc, starting at chapter 4, where Spivak introduces
formal category-theoretic concepts. Each section will refer to the appropriate
part of one of the Coq files.

\setcounter{chapter}{3}
\chapter{Basic Category Theory}


\section{Categories and Functors}

\subsection{Categories}

\reference{Definition of a Category}{Category.v}{Class Category}

\end{document}